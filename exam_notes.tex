\documentclass{report}
\usepackage[utf8]{inputenc}
\usepackage{amssymb}
\usepackage{amsmath}
\usepackage{algorithmicx}
\usepackage[ruled]{algorithm}
\usepackage[noend]{algpseudocode}

\usepackage{graphicx}

\title{Advanced Algorithms and Datastructures - Exam Notes}
\author{André O. Andersen}
\date{2021}

\begin{document}

\maketitle

\section*{Max Flow}
\subsection*{Disposition}
\begin{enumerate}
    \item Introduction
    \item Example of running the Edmonds-Karp Algorithm
    \item Proof of the \textit{Max-flow min-cut theorem}
\end{enumerate}
\begin{center}
    \begin{figure}[h!]
        \centering
        \includegraphics[scale = 0.8]{entities/max_flow_example.PNG}
    \end{figure}
\end{center}

\clearpage
\subsection*{Presentation}
\clearpage

\section*{Linear Programming and Optimization}
\subsection*{Disposition}
\begin{enumerate}
    \item Introduction
    \item Preparing and running SIMPLEX on example
    \item Proof of weak duality
\end{enumerate}

\clearpage
\subsection*{Presentation}
\clearpage

\section*{Randomized Algorithms}
\subsection*{Disposition}
\begin{enumerate}
    \item Introduction
    \item Example on running randomized quicksort + motivation behind randomness
    \item Analysis of expected runtime of randomized quicksort
    \item Example on running the min-cut algorithm
    \item Las Vegas Algorithms vs Monte Carlo Algorithms
\end{enumerate}

\clearpage
\subsection*{Presentation}
\clearpage

\section*{Hashing}
\subsection*{Disposition}
\clearpage
\subsection*{Presentation}
\clearpage

\section*{Van Emde Boas Trees}
\subsection*{Disposition}
\clearpage
\subsection*{Presentation}
\clearpage

\section*{NP-Completeness}
\subsection*{Disposition}
\clearpage
\subsection*{Presentation}
\clearpage

\section*{Exact Exponential Algorithms and Parameterized Complexity}
\subsection*{Disposition}
\clearpage
\subsection*{Presentation}
\clearpage

\section*{Approximation Algorithms}
\subsection*{Disposition}
\begin{enumerate}
    \item Introduction
    \item Definition of the \textit{approximation ratio}, a $\rho(n)$\textit{-approximation algorithm} and a \textit{randomized} $\rho(n)$\textit{-approximation algorithm}.
    \item The Vertex-cover problem
    \begin{enumerate}
        \item Introduction
        \item Proof that \texttt{APPROX-VERTEX-COVER} is a 2-approximation algorithm
    \end{enumerate}
    \item MAX-3-CNF
    \begin{enumerate}
        \item Introduction
        \item Proof that the randomized algorithm for \texttt{MAX-3-CNF} is a randomized $8/7$-approximation algorithm
    \end{enumerate}
\end{enumerate}

\clearpage
\subsection*{Presentation}
\textbf{Definition of \textit{approximation ratio}} \\
We say that an algorithm for a problem has an \textit{\textbf{approximation ratio}} of $\rho(n)$ if, for any input of size $n$, the cost $C$ of the solution produced by the algorithm is within a factor of $\rho(n)$ of the cost $C^*$ of an optimal solution
$$
\max \left( \frac{C}{C^*}, \frac{C^*}{C} \right) \leq \rho(n).
$$
\textbf{Definition of \textit{$\rho(n)$-approximation algorithm}} \\
If an algorithm achieves an approximation ratio of $\rho(n)$, we call it a \textit{\textbf{$\rho(n)$-approximation algorithm}}.
\\
\\
\textbf{Definition of \textit{randomized} $\rho(n)$\textit{-approximation algorithm}} \\
We say that a randomized algorithm for a problem has an \textit{\textbf{approximation ratio}} of $\rho(n)$ if, for any input of size $n$, the expected cost $C$ of the solution procuded by the randomized algorithm is within a factor of $\rho(n)$ of the cost $C^*$ of an optimal solution:
$$
\max \left( \frac{C}{C^*}, \frac{C^*}{C} \right) \leq \rho(n).
$$
We call a randomized algorithm that achieves an approximation ratio of $\rho(n)$ a \textit{\textbf{randomized $\rho(n)$-approximation algorithm}}
\\
\\
\textbf{Introduction to \textit{vertex cover}} \\
\noindent A \textit{\textbf{vertex cover}} of an undirected graph $G = (V, E)$ is a subset $V' \subseteq V$ such that if $(u, v)$ is an edge of $G$, then either $u \in V'$ or $v \in V'$ (or both). The size of a vertex cover is the number of vertices in it. The \textit{\textbf{vertex-cover problem}} is to find a vertex cover of minimum size in a given undirected graph. We call such a vertex cover an \textit{\textbf{optimal vertex cover}}. 
\\
\\
\noindent The set $C$ of vertices that is returned by \texttt{APPROX-VERTEX-COVER} is a vertex cover, since the laogrithm loops until every edge in $G.E$ has been covered by some vertex in $C$.
\begin{algorithm}[htbp]
    \caption{APPROX-VERTEX-COVER}
    \begin{algorithmic}[1]
        \Require Undirected graph $G$
        \State $C = \emptyset$
        \State $E' = G.E$
        \While{$E' \neq \emptyset$}
            \State let $(u, v)$ be an arbitrary edge of $E'$
            \State $C = C \cup \{u, v\}$
            \State remove from $E'$ edge $(u, v)$ and every edge incident on either $u$ or $v$
        \EndWhile
        \State \textbf{return} $C$
    \end{algorithmic}
\end{algorithm}

\noindent \textbf{Proof that \texttt{APPROX-VERTEX-COVER} is a 2-approximation algorithm} \\
Let $A$ denote the set of edges that line $4$ picked. Not two edges in $A$ share an endpoint. Thus no two edges in $A$ are covered by the same vertex from an optimal cover $C^*$, and we have the lower bound
\begin{equation}
    \label{eqn:vertex_cover_optimal_lower_bound}
    |C^*| \geq |A|
\end{equation}
on the size of an optimal vertex cover. Since $A$ consists of the edges between two vertices in $C$ (and since all of the elements in $C$ are unique), we have the (exact) upper bound on the size of the vertex cover returned
\begin{equation}
    \label{eqn:vertex_cover_returned_upper_bound}
    |C| = 2|A|
\end{equation}
Combining equation (\ref{eqn:vertex_cover_optimal_lower_bound}) and (\ref{eqn:vertex_cover_returned_upper_bound}), we obtain
$$|C| = 2|A| \leq 2|C^*|$$

% MANGLER INTRODUCTION, MAX-3-CNF OG MÅSKE ET KORT EKSEMPEL PÅ ET VERTEX COVER

\clearpage

\section*{Polygon Triangulation}
\subsection*{Disposition}
\begin{enumerate}
    \item Introduction
    \item The 3-coloring approach
    \begin{enumerate}
        \item Example on running the algorithm
        \item Proving that the 3-coloring approach is optimal in worst case
    \end{enumerate}
    \item Example on partitioning a polygon into monotone pieces + runtime analysis
    \item Example on triangulating a monotone polygon + runtime analysis
\end{enumerate}

\clearpage
\subsection*{Presentation}
\clearpage

\end{document}